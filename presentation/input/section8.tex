% !TEX root = ../document.tex
% !TeX spellcheck = pt_BR

\section{Resultados}

\begin{frame}{Resultados}
    
\end{frame}
