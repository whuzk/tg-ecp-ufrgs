% !TEX root = ../document.tex
% !TeX spellcheck = pt_BR

\section{Proposta de trabalho}
\label{sec:section5}
Até aqui tratamos da implementação das etapas iniciais dos algoritmos -- preprocessamento, extração de características, preparação dos dados --, mas não entramos em questões de classificação. Este será o próximo passo no desenvolvimento do trabalho. Abaixo é apresentada uma proposta de implementação, seguida de um roteiro para validação dos resultados e, por último, um cronograma preliminar de atividades.

\subsection{Implementação}
No caso do método de Rocha et al. \ldots

No caso do método de Mohebbi e Moghadam \ldots

No caso do método de Gopalakrishnan et al. \ldots

\subsection{Testes e Validação}
Para avaliar a confiabilidade dos algoritmos, lançaremos mão de métricas definidas em termos no número de verdadeiros positivos (VP), verdadeiros negativos (VN), falsos positivos (FP) e falsos negativos (FN) obtidos no teste. Estes conceitos são melhor visualizados numa matriz de confusão, conforme ilustra a figura ?.

figura ou tabela\ldots

Em especial, estamos interessados na sensibilidade ($S_e$) e na preditividade positiva ($P_p$). A sensibilidade pode ser entendida como a probabilidade do teste detectar batidas isquêmicas, enquanto a preditividade positiva pode ser interpretada como uma medida do ``otimismo'' do teste com respeito ao conjunto de batidas analisadas. Seus valores podem ser obtidos pelas expressões abaixo:
\begin{equation} \label{equ:metrics}
    S_e = \frac{VP}{VP+FN}
    \quad\quad
    P_p = \frac{VP}{FP+VP}
\end{equation}

De modo geral, estes conceitos são usados para avaliar o desempenho de um teste diagnóstico frente a um método de referência. Este método alternativo deve ser confiável e, preferencialmente, realizado por especialistas. Em nosso caso, teremos o diagnóstico dado por cardiologistas que, de maneira unânime, classificaram todas as batidas de todos os registros de ECG num determinado banco de dados. Os bancos utilizados são o European ST-T e o Long-Term ST. Ambos os bancos, independentemente do formato de dados, possuem um campo que indica o diagnóstico das batidas. Neste campo, a letra ? indica inversão da onda T, enquanto a letra ? indica elevação/depressão do segmento ST.

Ao final dos testes, obteremos as medidas necessárias e as compararemos com o resultado publicado nos artigos originais de cada método. Mais importante ainda, compararemos os métodos entre si, e tomaremos uma decisão quanto ao melhor método para ser devidamente implementado e implantado num sistema móvel de monitoramento cardíaco.

Num segundo plano, tentaremos avaliar a eficiência dos algoritmos em termos do tempo de execução. Para tanto, será necessário simular uma configuração de tempo-real, em que as amostras do sinal de ECG são repassadas à entrada do algoritmo uma a uma. Os algoritmos deverão então ser adaptados para armazenar amostras recentes do sinal e descartar a mais antiga quando uma nova é recebida. Dessa forma, trabalhar-se-á com uma janela temporal do sinal de entrada, e uma nova iteração do algoritmo se realizará a cada nova amostra. Entretanto, deve-se salientar que esta não é uma métrica precisa, na medida em que o método escolhido deverá atuar num dispositivo móvel de uso geral. Isto significa que, dependendo  do hardware do dispositivo, um método poderia executar mais rapidamente do que outro, mesmo que a simulação forneça resultado contraditório.


\subsection{Cronograma}
Texto\ldots
