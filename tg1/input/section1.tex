% !TEX root = ../document.tex
% !TeX spellcheck = pt_BR

\section{Visão geral}
\label{sec:section1}

A escolha dos algoritmos é resultado de um estudo realizado por Guilherme Lima \cite{Lima01}, em que foram analisados diversos artigos científicos sobre a detecção de isquemia cardíaca em ECGs. Lima coletou dados de sensibilidade, preditividade positiva e acurácia, e selecionou 5 métodos como candidatos para uma investigação mais profunda, dos quais 3 estão sendo implementados e serão discutidos neste trabalho.

O primeiro método é o trabalho proposto por Rocha et al. \cite{Rocha10}, cuja estratégia de detecção envolve a caracterização do complexo QRS, do segmento ST e da onda T. Esta caracterização é baseada em uma análise tempo-frequencial do sinal de ECG e também na expansão em funções de Hermite (discutidas em detalhe na próxima seção). O algoritmo consiste em extrair dois conjuntos de características para cada batida, e usar um comitê de redes neurais para classificar uma batida em isquêmica ou não-isquêmica.

O segundo método é o de Mohebbi e Moghadam \cite{Mohebbi07}, em que a detecção é baseada na construção de um modelo (ou \emph{template}, em inglês) para o segmento ST de batidas normais. O algoritmo consiste basicamente em medir o quão diferente é um segmento ST em relação ao modelo, e fornecer essa diferença como entrada em uma rede neural. A rede então classificará a batida de que foi extraído o segmento em isquêmica ou não.

O terceiro método é o de Gopalakrishnan et al. \cite{Gopalak04}. Neste trabalho, uma estratégia semelhante à do primeiro método é adotada, usando funções de Hermite para aproximar o formato de uma batida cardíaca. Contudo, a formulação matemática aqui é mais forte, e os autores conseguem derivar um método de cálculo eficiente para as funções de Hermite, envolvendo matriz de Fourier e diversas propriedades da álgebra linear, como tridiagonalidade e comutatividade de matrizes, autovalores e autovetores, entre outras.