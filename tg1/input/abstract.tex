% !TEX root = ../document.tex
% !TeX spellcheck = pt_BR

\begin{abstract}
    This paper presents a comparative study on three algorithms for detection of myocardial ischemia in electrocardiograms. Parts of the algorithms e.g. preprocessing and feature extraction were implemented in MATLAB code and will be discussed here, while later parts e.g. classification using neural networks and validation will be addressed in a sequel work. The purpose of this study is to select the method that best suits a cardiac monitoring system with support for medical decision-making. The evaluation of the method's adequateness relies on speed, reliability and simplicity of implementation. This study is motivated by the desire to diagnose and treat myocardial ischemia, the most prevalent heart disease, thereby helping the greatest number of people.
\end{abstract}

\begin{resumo}
    Este artigo apresenta um estudo comparativo sobre três algoritmos de detecção de isquemia cardíaca em eletrocardiogramas. As primeiras etapas de cada algoritmo, como pré-processamento e extração de características, foram implementadas no MATLAB e serão discutidas a seguir, enquanto as últimas etapas, como classificação usando redes neurais e validação, serão desenvolvidas num próximo trabalho. O objetivo deste estudo é selecionar o método que melhor se adeque a um sistema de monitoramento cardíaco com suporte à tomada de decisão médica. A avaliação do método se dará pela análise de tempo, confiabilidade e simplicidade de implementação. A motivação para tal estudo é o desejo de diagnosticar e tratar a doença cardíaca mais comum, ajudando assim o maior número possível de pessoas.
\end{resumo}
