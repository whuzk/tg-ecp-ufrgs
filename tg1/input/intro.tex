% !TEX root = ../document.tex
% !TeX spellcheck = pt_BR

\section*{Introdução}
\label{sec:intro}

Doenças cardíacas são a principal causa de morte no Brasil e no mundo. De acordo com a Organização Mundial da Saúde \cite{Who04}, elas representam cerca de 30\% de todas as mortes mundiais. Em especial, a doença arterial coronariana é responsável por 12\% das mortes (7.2 milhões), tornando-a a principal causa de morte dentre as cardiopatias.

Esta doença provoca uma condição chamada isquemia do miocárdio, ou isquemia cardíaca, que é a falta de suprimento de sangue no músculo do coração. Eventos isquêmicos podem ser detectados em exames eletrocardiográficos (ECGs) pela análise de ondas características como o complexo QRS, o segmento ST e a onda T. Batidas isquêmicas são distinguíveis no ECG pois alteram o formato dessas ondas, sendo que a elevação ou depressão do segmento ST, bem como a inversão da onda T, sãs os principais critérios sobre os quais especialistas baseiam seus diagnósticos de isquemia cardíaca.

Na busca de acelerar o processo diagnóstico e permitir que pacientes recebam tratamento adequado o mais rapidamente possível, há um esforço da comunidade científica em desenvolver sistemas móveis capazes de detectar isquemia \emph{in loco}. Sob esta perspectiva, tentaremos avaliar alguns métodos de detecção de isquemia já existentes no estado-da-arte, e selecionar aquele que apresenta melhor desempenho para ser empregado num sistema de monitoramento cardíaco móvel.

O trabalho está organizado da seguinte maneira: a seção \ref{sec:section1} dá uma visão geral dos métodos e de como eles foram selecionados; as seções \ref{sec:section2} a \ref{sec:section4} apresentam a implementação das primeiras etapas de cada um dos três algoritmos; a seção \ref{sec:section5} mostra os resultados e faz uma comparação dos algoritmos em termos de complexidade; por fim, a seção \ref{sec:section6} apresenta uma proposta de trabalho a ser desenvolvido para dar sequência ao presente estudo.
