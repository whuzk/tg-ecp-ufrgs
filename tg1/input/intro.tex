%!TEX root = ../document.tex

\section*{Introduction}
\addcontentsline{toc}{section}{Introduction}
\label{sec:intro}

Cardiovascular diseases are the main cause of death worldwide. According to the World Health Organization \cite{Who04}, they represented about 30\% of all global deaths in 2004. Ischemic heart disease -- also called coronary artery disease -- was responsible for 12\% of the deaths (7.2 million), which makes it the leading cause of death and, thus, the focus of international research effort.

This cardiac condition is originated by atherosclerosis, the thickening of artery walls by accumulation of fatty substances. It reduces the blood flow to the heart, and originates a disorder called myocardial ischemia -- \textit{ischemia} comes from Greek and signifies ``blood retention''. Myocardial ischemia may induce chest pain, known as \textit{angina pectoris}, or may have no symptoms at all. Since blood is responsible for carrying oxygen and removing metabolic waste, a prolonged deprivation of blood supply to the heart can lead to cellular necrosis -- premature death of cells in living tissue --, which in turn causes myocardial infarction.

Ischemic episodes can be detected in an electrocardiogram (ECG) by analyzing its characteristic waves, namely the QRS complex, ST segment and T wave (section 1 introduces briefly these concepts). Distinguishable artifacts, such as ST depression and T wave inversion, are the main criteria on which specialists base their diagnosis for ischemia. Thus, many methodologies for automatically detecting ischemic episodes in ECGs have been proposed that take into account these and other criteria.

The research effort seeks to aid the medical community in diagnosing myocardial ischemia at early stages. With computer-assisted detection of ischemic episodes, physicians are able to identify the disease more quickly and make adequate decisions (e.g. reestablishing blood supply to the heart by means of surgery). That is why, in this work, we attempt to assess a few existing techniques and learn what advantages they have with respect to one another, and, especially, if they can be deployed in a real cardiac monitoring system with support for medical decision-making.

The choice of algorithms to be discussed here is the result of a previous study realized by Guilherme Lima \cite{Lima01}. In the aforementioned study, several methods proposed by various authors in scientific articles were investigated and compared, based primarily on positive predictivity and sensibility, but also accuracy when this information was available or computable\footnote{Sensibility, specificity, positive and negative predictivity, and accuracy, are concepts used to determine the degree to which a diagnostic test (that needs to be assessed) can yield true or false results, with respect to a well-known reference method (usually performed by specialists).}. Lima selected five algorithms as candidates for deeper investigation, three of which were chosen to be implemented in the present work.

The arrangement of this paper is as follows: the first chapter gives a short description of how the heart works, as well as an overview of ECG signal acquisition; the second, third and fourth chapters address the implementation of the algorithms proposed by Rocha et al. \cite{Rocha10},  Mohebbi et al. \cite{Mohebbi07} and Gopalakrishnan et al. \cite{Gopalak04}, respectively; the final chapter compares the three methods and gives conclusion.
