% !TEX root = ../document.tex
% !TeX spellcheck = pt_BR

% ---
% Folha de rosto
% (o * indica que haverá a ficha bibliográfica)
% ---
%\imprimirfolhaderosto*
% ---

% ---
% Inserir a ficha bibliografica
% ---

% Isto é um exemplo de Ficha Catalográfica, ou ``Dados internacionais de
% catalogação-na-publicação''. Você pode utilizar este modelo como referência. 
% Porém, provavelmente a biblioteca da sua universidade lhe fornecerá um PDF
% com a ficha catalográfica definitiva após a defesa do trabalho. Quando estiver
% com o documento, salve-o como PDF no diretório do seu projeto e substitua todo
% o conteúdo de implementação deste arquivo pelo comando abaixo:
%
% \begin{fichacatalografica}
%     \includepdf{fig_ficha_catalografica.pdf}
% \end{fichacatalografica}
% ---

% ---

% ---
% Inserir folha de aprovação
% ---

% Isto é um exemplo de Folha de aprovação, elemento obrigatório da NBR
% 14724/2011 (seção 4.2.1.3). Você pode utilizar este modelo até a aprovação
% do trabalho. Após isso, substitua todo o conteúdo deste arquivo por uma
% imagem da página assinada pela banca com o comando abaixo:
%
% \includepdf{folhadeaprovacao_final.pdf}

% ---
% Agradecimentos
% ---
% \begin{agradecimentos}
% Os agradecimentos principais são direcionados à Gerald Weber, Miguel Frasson,
% Leslie H. Watter, Bruno Parente Lima, Flávio de Vasconcellos Corrêa, Otavio Real
% Salvador, Renato Machnievscz\footnote{Os nomes dos integrantes do primeiro
% projeto abn\TeX\ foram extraídos de
% \url{http://codigolivre.org.br/projects/abntex/}} e todos aqueles que
% contribuíram para que a produção de trabalhos acadêmicos conforme
% as normas ABNT com \LaTeX\ fosse possível.

% Agradecimentos especiais são direcionados ao Centro de Pesquisa em Arquitetura
% da Informação\footnote{\url{http://www.cpai.unb.br/}} da Universidade de
% Brasília (CPAI), ao grupo de usuários
% \emph{latex-br}\footnote{\url{http://groups.google.com/group/latex-br}} e aos
% novos voluntários do grupo
% \emph{\abnTeX}\footnote{\url{http://groups.google.com/group/abntex2} e
% \url{http://abntex2.googlecode.com/}}~que contribuíram e que ainda
% contribuirão para a evolução do \abnTeX.

% \end{agradecimentos}
% ---

% ---
% Epígrafe
% ---
% \begin{epigrafe}
%     \vspace*{\fill}
%     \begin{flushright}
%         \textit{``Não vos amoldeis às estruturas deste mundo, \\
%         mas transformai-vos pela renovação da mente, \\
%         a fim de distinguir qual é a vontade de Deus: \\
%         o que é bom, o que Lhe é agradável, o que é perfeito.\\
%         (Bíblia Sagrada, Romanos 12, 2)}
%     \end{flushright}
% \end{epigrafe}
% % ---