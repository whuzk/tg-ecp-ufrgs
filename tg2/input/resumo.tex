% !TEX root = ../document.tex
% !TeX spellcheck = pt_BR

\begin{resumo}

\addcontentsline{toc}{chapter}{Resumo}
\setlength{\parindent}{0.6cm}

Esta monografia apresenta um estudo comparativo prático de três algoritmos de detecção de isquemia cardíaca em eletrocardiogramas. Todas as etapas de cada algoritmo -- pré-processamento, extração de características e classificação --, são implementadas em MATLAB e na linguagem C.

O trabalho se propõe a atingir dois objetivos: primeiramente corroborar ou refutar os valores de medidas de confiabilidade, como sensibilidade e valor preditivo positivo, obtidos pelos autores dos métodos em sua proposta original; e em segundo lugar determinar o melhor método para ser implantado num sistema de monitoramento cardíaco com suporte à tomada de decisão médica. Isto é, selecionar dentre os métodos aquele que for mais confiável.

Este trabalho foi realizado em conjunto com o mestrado de Guilherme Lazarotto de Lima, Mestrando em Computação pela UFRGS.

\null\vfill
\textbf{Palavras-chaves}: biomedicina. processamento de sinais. eletrocardiograma. isquemia cardíaca. redes neurais. monitoramento remoto.

\end{resumo}