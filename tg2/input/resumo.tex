% !TEX root = ../document.tex
% !TeX spellcheck = pt_BR

\begin{resumo}

\addcontentsline{toc}{chapter}{Resumo}
\setlength{\parindent}{0.6cm}

Esta monografia apresenta um estudo comparativo prático de três algoritmos de detecção de isquemia cardíaca em eletrocardiogramas. Todas as etapas de cada algoritmo -- pré-processamento, extração de características e classificação --, são implementadas em MATLAB e nas linguagens C e C++.

O trabalho se propõe a atingir os seguintes objetivos: primeiramente inspirar-se nos algoritmos originais para criar uma implementação própria; em segundo lugar, pensa-se em unificar a implementação de modo que apenas a parte de extração (aquela que realmente define cada método) seja diferente; em terceiro lugar, quer-se avaliar o desempenho de cada método através de métricas de confiabilidade, como a sensibilidade e a preditividade positiva; por fim, deseja-se eleger o método que se sobressai como o mais confiável em termos destas métricas.

A razão de selecionar o método mais confiável é que, num momento futuro, poder-se-ia implementar o método de verdade num dispositivo do tipo \emph{smartphone} ou mesmo num sistema de monitoramento cardíaco com suporte à tomada de decisão médica.

Este trabalho foi realizado em conjunto com o mestrado de Guilherme Lazarotto de Lima, Mestrando em Computação pela UFRGS.

\null\vfill
\textbf{Palavras-chaves}: biomedicina. processamento de sinais. eletrocardiograma. isquemia cardíaca. redes neurais. monitoramento remoto.

\end{resumo}