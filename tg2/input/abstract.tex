% !TEX root = ../document.tex
% !TeX spellcheck = en_US

\begin{resumo}[Abstract]
\begin{otherlanguage*}{english}

\addcontentsline{toc}{chapter}{Abstract}
\setlength{\parindent}{0.6cm}

This monograph presents a practical, comparative study of three algorithms for detection of myocardial ischemia in electrocardiograms. All stages in each of the algorithms -- preprocessing, features extraction and classification --, are implemented in MATLAB and in the C and C++ programming languages.

The purpose of this work is to achieve the following goals: firstly, to draw on the original algorithms to create a similar, but own implementation; secondly, to unify the implementation in a manner such that only the extraction phase (the one that really defines each method) be different; then, it wishes to evaluate the performance of the methods through a reliability analysis, taking into account metrics like sensitivity and positive predictivity; ultimately, it is desired that the method which out-tops the others in terms of these metrics be elected as the most reliable.

The reason to select the most reliable method is that, in a future time, it could be effectively implemented in a \emph{smartphone} or even in a cardiac monitoring system with support for medical decision-making.

This work was done in cooperation with Guilherme Lazarotto de Lima, in his research for the Master's degree in Computer Science at UFRGS.


\null\vfill
\textbf{Key-words}: biomedicine. signal processing. electrocardiogram. myocardial ischemia. neural networks. telehomecare.

\end{otherlanguage*}
\end{resumo}