% !TEX root = ../document.tex
% !TeX spellcheck = en_US

\begin{resumo}[Abstract]
\begin{otherlanguage*}{english}

\addcontentsline{toc}{chapter}{Abstract}
\setlength{\parindent}{0.6cm}

This monograph presents a practical, comparative study of three algorithms for detection of myocardial ischemia in electrocardiograms. All stages in each of the algorithms -- preprocessing, features extraction and classification --, are implemented in MATLAB and in C language.

The purpose of this work is twofold: firstly, it wishes to corroborate or refute the values of reliability measures, such as sensitivity and positive predictive value, attained by the authors of the methods in their original papers; secondly, it seeks to ascertain the best method to incorporate a cardiac monitoring system with support for medical decision-making. In other words, to elect among the methods the one which is most reliable.

This work was done in cooperation with Guilherme Lazarotto de Lima, in his research for the Master's degree in Computer Science at UFRGS.


\null\vfill
\textbf{Key-words}: biomedicine. signal processing. electrocardiogram. myocardial ischemia. neural networks. telehomecare.

\end{otherlanguage*}
\end{resumo}